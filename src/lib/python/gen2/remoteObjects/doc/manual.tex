\documentclass[11pt]{report}

% sensible specification of margins, etc.
\usepackage[paperwidth=8.5in,paperheight=11in]{geometry}
\geometry{inner=1.0in, outer=1.5in, top=1.0in, bottom=0.5in,
  includefoot}
\geometry{twoside}

% for displaying graphics
\usepackage[]{graphicx}

% use modern and attractive fonts
\usepackage{fontspec}
\setmainfont[Mapping=tex-text]{Times New Roman}
\setsansfont[Mapping=tex-text]{Arial}
%\setmonofont[Scale=0.85]{Courier New}
\setmonofont[]{Courier New}
%% \setmainfont[Mapping=tex-text]{Bitstream Vera Serif}
%% \setsansfont[Mapping=tex-text]{Bitstream Vera Sans}
%% \setmonofont[Scale=0.85]{Bitstream Vera Sans Mono}

% Save space in headings
%% \usepackage[small,compact]{titlesec}
%% \titlespacing{\section}{0pt}{*0}{*0}
%% \titlespacing{\subsection}{0pt}{*0}{*0}
%% \titlespacing{\subsubsection}{0pt}{*0}{*0}

% provides itemize* environment
\usepackage{mdwlist}

\usepackage{verbatim}
\newenvironment{myverbatim}
  {\setlength{\parskip}{0pt}\verbatim}
  {\endverbatim}

%\raggedright

\setlength{\parindent}{0cm}
\setlength{\parskip}{1em plus0.5em minus0.5em}

\setlength{\marginparsep}{0.35in}
\setlength{\marginparwidth}{1.5in}

\title{Subaru Remote Objects Middleware} 
\author{Eric Jeschke\\
{\tt eric@naoj.org}\\
\\
Observation Control Software Group\\
Subaru Telescope\\
National Astronomical Observatory of Japan}

\pagestyle{plain}    

\begin{document} 
\maketitle 

\begin{abstract}
This is a reference manual for software developers using the Subaru
Remote Objects middleware (hereafter referred to as ``SRO'').  SRO
provides a fairly simple to use infrastructure based on top of the XML-RPC
standard.  While XML-RPC provides a robust dynamic remote procedure call
mechanism, it does not specify some common upper layer services that are
used by many middleware systems, such as a object name service, service
failover, publish/subscribe system, etc.  SRO attempts to fill in these
services as simply, inobtrusively and transparently as possible.
\end{abstract}

\tableofcontents
\setcounter{tocdepth}{3}

\newpage

%%%%%%%%%%%%%%%%%%%%%%%%%%%%%%%%%%%%%%%%%%%%%%%%%%%%%%%%%%%%% 
\chapter{Introduction}
This is a reference manual for software developers using the Subaru
Remote Objects Middleware (hereafter referred to as ``SRO'').  SRO
provides a fairly simple to use middleware based on top of the XML-RPC
standard.  While XML-RPC provides a robust dynamic remote procedure call
mechanism, it does not specify some common upper layer services that are
used by many middleware systems, such as a object name service, service
failover, publish/subscribe system, etc.  SRO attempts to fill in these
services as simply, inobtrusively and transparently as possible.
SRO provides:
\begin{itemize}
\item a simplified API for creating remote object services and
  remote object proxies to those services;
\item provision for automatic failover--a proxy can redirect
  automatically to another instance of a service in the case of failure;
\item a simple name service based on a flat namespace of strings, that
  allows host independent location of services (e.g. on a cluster);
\item a simple publish-subscribe (hereafter referred to as ``pub/sub'')
  service based on the idea of channels that is very scalable, and
  allows things like publication or subscription to an aggregation of
  channels. 
\item a simple ``manager'' service for starting up and managing
  processes on a group of hosts.
\end{itemize}

\section{History of SRO}
SRO was initially developed as a way to abstract a middleware system.
The Subaru Telescope Observation System developers knew they needed to
use a middleware system, but spent a long time trying to figure out
which middleware system to use: CORBA, ICE, SOAP, etc.  All of
them seemed to have serious drawbacks.

A decision of which middleware to use is a significant one and
practically speaking paralyzes development until the choice is made,
because it is the foundation of everything that will be built on top.
In an effort to break the paralysis, one developer decided to try and
write a wrapper to encapsulate the middleware in a way that aligned with
the ``ideal'' remote object service--in essence, to write the ideal
remote objects API.  The attributes of such a system might include: 
\begin{itemize}
\item it should be an ``object'' based system, exporting the methods of
  objects residing on local or remote nodes;
\item it should be as transparent as possible, compared to using
  ``local'' objects--it should support a wide range of useful data types
  without special marshalling calls;
\item it should be cross-platform, allowing services to interoperate
  between languages and operating systems;
\item it should support a name service that allows remote objects to
  live anywhere on the network, and allow one to look up an object or
  service by name;
\item it should be as uninvasive as possible, as far as boilerplate code
  addition to other code using it (this extends to writing method
  signature stubs--python is a dynamically typed language);
\item failure handling is important--it should have a way to ``fail
  over'' to alternate objects/services without any extra code.
\end{itemize}

Since XML-RPC was available in the standard Python installation it
was chosen as the transport mechanism to actually implement the method
calls behind a proxy/service object encapsulation.  What the developers
found was that the result was a very effective system that matched the
above attributes.  After many years of use the XML-RPC base has not been
replaced with another transport mechanism.

Due to its XML-RPC underpinnings, SRO has some significant limits on
performance compared to ``binary on the wire'' encodings, but for many
uses it is surprisingly fast and effective.  The use of http as the
underlying transport means that there are some significant advantages as
well.  For example, it is very easy to tunnel through a firewall, to
encrypt (using https) or to use authentication (via simple http
authentication).  It is also completely dynamic, requiring no method
stub files to be generated for most dynamic languages.  

\chapter{SRO Basics}
SRO was implemented in the Python programming language.  Because it uses
XML-RPC as its transport layer, it is effectively cross-platform to any
other language that can speak XML-RPC.  However, the simple API to
creating and using SRO is most fully realized in Python, and other
language bindings are few and minimally supported at the current time.
Please see Appendix \ref{app:otherlang} for details.  These examples
will all be shown in Python.

\section{Initializing SRO}
Every middleware has a certain amount of boilerplate code that must be
set up in order to use its features.  SRO tries to make this as minimal
as possible.  For both client and server side use, the simple expression
\begin{verbatim}
import remoteObjects as ro
...
ro.init()
\end{verbatim}
is sufficient in most cases to initialize the program to subsequently
create remote object servers and proxies.  
In the case where the host is not running the SRO name service daemon
then it is usually necessary to specify the name of a host or hosts
which are running the name service, so that services can be advertised
or looked up.  This can be done by specifying the name of the host as a
parameter to {\tt ro.init()}:
\begin{verbatim}
import remoteObjects as ro
...
rohosts = ['kiwi', 'banana']

ro.init(rohosts)
\end{verbatim}
alternatively, if no parameter is given and the environment variable
{\tt RO\_HOSTS} (a comma separated list of hosts) is set, it will be
used.
The order of specification is:
\begin{itemize}
\item if a list of hosts is given, it will be used;
\item if the environment variable {\tt RO\_HOSTS} is set if will be used;
\item if there is a name server running on the local host it will be
  consulted for the host list;
\item otherwise the node is considered to be an isolated node with no
  functioning name service.
\end{itemize}

The SRO interface only needs to be initialized once prior to any
subsequent calls that attempt to use it.

\section{Creating a SRO Proxy Object}
Creating proxies to SRO services is very simple.  Assuming that the
above code preamble has been executed already, the following code
fragment will create a proxy object to a remote service:
\begin{verbatim}
catalog = ro.remoteObjectProxy('catalog')
\end{verbatim}
As you can see, services are specified by simple strings in a flat
namespace.  The client does not know on which host(s) the service is running
or what port is being used for communication.

This call would set up a proxy object to the ``catalog'' service.
With some significant caveats, the proxy can be treated more or less like a
local object.  In the usual case, method calls may be used with it.
For example, the following call might return a list of tuples of
items matching the particular {\tt part\_no}:
\begin{verbatim}
results = catalog.search(part_no)
\end{verbatim}
There are important differences between a local python object and a
remote proxy object that are due to the obvious fact that the remote
object is running in a separate process, possibly on a different host. 
We can succinctly summarize these differences as follows:
\begin{itemize}
\item Direct attribute access is generally not supported, except to
  ``virtual methods'' provided by the service.  In other words, the
  remote interface is encapsulated via method calls (this is almost
  always a ``good thing'' when it comes to remote services).
\item Only certain data types may be passed as parameters to the remote
  methods.  Fortunately these data types include most of the rich native
  data types that Python supports.  These types are summarized in
  Appendix \ref{app:datatypes}. 
\item Parameter passing is by copy, not by reference, so that it does
  not make sense to pass mutable structures and expect that any changes
  to the structure on the remote side will be reflected in the local
  side.  For example, we cannot pass a dictionary as a parameter and
  expect additions to it to be reflected on our copy.  (We can pass a
  dictionary as a parameter and receive a dictionary as part of a method
  return value.)
\item Keyword parameters cannot be passed by keyword, only by position.
  This is again posed by the limitation of the underlying transport
  mechanism. 
\end{itemize}

\section{Creating a SRO Server Object}

\chapter{Implementing an SRO System}
In keeping with its minimalist philosophy, the only thing really
necessary to creating an SRO system is to run the name service daemon
somewhere and point all of the hosts at it using the {\tt ro.init(host)}
initializer.  The name service daemon comes with the SRO package and can
be run as in the following example:
\begin{verbatim}
$ remoteObjectNameServer.py --loglevel=info --log=ro_names.log
\end{verbatim}
use the {\tt --help} option to see all of the possible parameters.

This scenario will work perfectly fine for most use cases.  However, SRO
has some useful failover capabilites built into it, including the
ability to handle failover of the daemon services themselves.  If you
are implementing services on a high availability cluster, for example, a
practical method is to run the name service daemon on each node of the
cluster.  In that case you can run the daemon on each node.
In this sort of scenario the SRO name service daemons use the SRO
pub/sub system to communicate between themselves about the services that
each knows about.  The pub/sub daemon is run as follows

\begin{verbatim}
$ PubSub.py --loglevel=info --log=ro_pubsub.log
\end{verbatim}

%%%%%%%%%%%%%%%%%%%%%%%%%%%%%%%%%%%%%%%%%%%%%%%%%%%%
\appendix    %>>>> this command starts appendixes
%%%%%%%%%%%%%%%%%%%%%%%%%%%%%%%%%%%%%%%%%%%%%%%%%%%%

\label{app:datatypes}
\chapter{Data types that can be used with SRO}
SRO builds upon XML-RPC as a transport mechanism and only provides a
very thin layer on top.  Thus only XML-RPC compliant data types may be
marshalled between a proxy and the service providing it.  Fortunately
this is a rich data set that includes many of the most useful types.
The following table attempts to summarize the list (see XML-RPC
documentation for more details).

\label{app:advancedtopics}
\chapter{SRO Advanced Topics}

\section{Remote Object Clients}
In SRO terminology a \emph{remote object client} is the entity upon
which the remote object proxy is built.  A remote object client is a
proxy object that is set up by providing an actual hostname and port 

\label{app:otherlang}
\chapter{Using SRO with Languages other than Python}

\end{document} 
